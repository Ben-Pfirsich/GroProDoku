\chapter{Zusammenfassung und Ausblick}\label{ch:zusammenfassung-und-ausblick}

\section{Zusammenfassung}\label{sec:zusammenfassung}
Es wurde ein Programm entwickelt, welches die Verarbeitung von Signalen eines Autokorrelators wie gefordert umsetzt.

Eingabe, Verarbeitung und Ausgabe laufen kontinuierlich in voneinander unabhängigen Prozessen.
Datensätze werden mit einer Frequenz von 20 Hz, also für 0,05 Sekunden, von der Eingabe zur Verfügung gestellt.
Die Verarbeitung verarbeitet keinen Datensatz doppelt und das Programm endet, wenn alle Datensätze ausgegeben wurden.
Die Ergebnisse werden im erwünschten Format in Textdateien geschrieben, sodass sie z.B. in Python weiterverarbeitet werden können.

Zudem wurde beim Entwurf besonders auf Erweiterbarkeit geachtet:
Die Signaldatenquelle kann einfach ausgetauscht werden, indem das Eingabe-Interface mit einer neuen Strategie implementiert wird.
Ebenso kann das Ausgabeformat bei Bedarf mit geringem Aufwand gewechselt werden.

Die Verfahren und ihre mathematischen Details wurden ausführlich dokumentiert.
Die Wahl von unterschiedlichen Datenstrukturen wurde begründet und der Programmentwurf ist durch diverse UML- und Nassi-Shneiderman-Diagramme gut nachvollziehbar.
Für Entwickler wurde eine Dokumentation direkt im Source-Code erstellt.


\section{Ausblick}\label{sec:ausblick}
Das entwickelte Programm kann vielseitig erweitert und verbessert werden.
Neben der Ausbesserung von kleinen Fehlern und technischen Schulden, gibt es viel Raum zur Implementierung von neuer Funktionalität.

\subsection{Parallele Verarbeitung in den Komponenten}\label{subsec:parallel}
Da im originalen Entwurf schon Threads verwendet wurden, ist es naheliegend diesen Ansatz nochmal zu verfeinern.
Alle drei Komponenten könnten für die Bearbeitung von einzelne Datensätze jeweils einen eigenen Unterprozess starten.
Konkret kann dieses Verhalten in Java mit dem \enquote{ExecutorService} und Java Futures umgesetzt werden.
Besonders bei sehr großen Datensätzen könnte dadurch die Laufzeit des Programms verbessert werden.

\subsection{Refactoring}\label{subsec:refactoring}
Die Lesbarkeit des Codes ist an einigen Stellen verbesserungswürdig.

So wird zum Beispiel bei Fehlern in der Ein- und Ausgabe die gleiche Schnittstelle (\enquote{meldeDatVerarbeitet}), wie bei der erfolgreichen Ausgabe verwendet.
Das funktioniert, ist aber nicht besonders intuitiv.
In Zukunft könnte hierfür eine eigene Fehlerschnittstelle implementiert werden.
So könnten auch verbesserte Fehlerstatistiken gesammelt werden.

Im Entwurf wurde für einen einzelnen Messpunkt eine eigene Klasse (AKDatum) vorgesehen.
Diese hat aber in der aktuellen Implementierung keinen wirklichen Mehrwert, da nie auf einem einzelnen Datum gearbeitet wird, sondern immer nur auf dem Datensatz.
Man könnte also überlegen die Liste der Datenpunkte durch zwei Listen mit Double-Werten zu ersetzen.

\subsection{Logging und Monitoring}\label{subsec:log-moni}
Wenn die Anwendung in den produktiven Betrieb genommen wird, sollte unbedingt ein Konzept zum Logging und Monitoring entwickelt werden.
Die aktuelle Umsetzung schreibt Logging-Einträge auf die Konsole, was als \enquote{bad practice} angesehen wird.
Stattdessen sollte hier eine Logging-Framework verwendet werden, beispielsweise die \enquote{Simple Logging Facade for Java} (SLF4J), die viele unterschiedliche Implementierungen hat (unter anderem Log4J).

Um zudem eine bessere Übersicht über die internen Vorgänge des Programms zu gewinnen, könnten durch Monitoring-Tools (z.B. Micrometer: \url{https://micrometer.io/}) unterschiedliche Metriken definiert werden.
Diese könnte z.B. die Anzahl der verarbeiteten Datensätze, die Anzahl von Datensätzen innerhalb einer Komponente oder die Fehleranzahl in den Programmteilen dokumentieren.

\subsection{Neue Eingabestrategie}\label{subsec:eingabestrat}
In der Aufgabenstellung wurde angedeutet, dass sich die Art, wie Datensätze beim Programm ankommen, im produktiven Betrieb ändert.
Dafür kann beispielsweise eine neue Eingabestrategie implementiert werden, die über eine Web-Schnittstelle Datensätze in Empfang nimmt.
Eine SOAP-Schnittstelle, die XML-Daten entgegennimmt, wäre hier eine mögliche Lösung.
