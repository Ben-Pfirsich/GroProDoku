\chapter{Abweichung und Ergänzung}\label{ch:abweichung-und-ergaenzung}
Der grundlegende Entwurf hat sich nicht verändert, in der originalen Interpretation waren aber einige Fehler und Ungenauigkeiten, die ausgebessert werden mussten.
Diese werden nun beschrieben.

\section{Abweichungen}\label{sec:abweich}

\begin{itemize}[noitemsep]
    \item \textbf{Fehler beim Sonderfall zur Datenglättung:} An den Rändern der Liste mit den Datenpunkten musste bei der Glättung der x-Daten besonders aufgepasst werden, weil dort weniger Punkte für den gleitenden Mittelwert vorhanden sind.
    In der ersten Interpretation wurde fälschlicherweise davon ausgegangen, dass die Werte nicht nach den x-Werten sortiert vorliegen.
    Daher wurden für den gleitenden Mittelwert Punkte vom anderen Ende der Liste verwendet.
    In der tatsächlichen Implementierung ist dieser Fehler aufgefallen und wurde stattdessen wie in Abschnitt~\ref{subsec:glaettung} beschrieben umgesetzt.
    \item \textbf{Zähler für erfolgreiche Ausgabe:} Der Zähler, der der Indikator für das Programmende ist, wurde wie original geplant an zentraler Stelle in der Steuerung definiert.
    Im ursprünglichen Sequenzdiagramm wird dieser aber in der Ausgabe-Instanz verwaltet.
    Diese Inkonsistenz wurde ausgebessert.
    \item \textbf{Name von Methoden:} Weil bei der Planung viele Implementierungsdetails noch nicht bekannt waren, haben sich die Namen von vielen der Methoden im Vergleich zum ersten Klassendiagramm verändert (z.B. \enquote{run()} statt \enquote{starteVerarbeitung()}).
\end{itemize}

\section{Ergänzungen}\label{sec:erg}

\begin{itemize}
    \item \textbf{Schnittstellen zwischen den Komponenten:} Im originalen Klassendiagramm waren die Schnittstellen zwischen Eingabe, Verarbeitung und Ausgabe aus Zeitgründen nicht definiert.
    Diese wurden im Nachhinein genau spezifiziert.
    \item \textbf{Diverse Attribute und Hilfsmethoden:} Bei der Implementierung sind viele weitere Methoden und Attribute hinzugekommen, z.B. die Queues als Zwischenspeicher für Datensätze.
\end{itemize}
