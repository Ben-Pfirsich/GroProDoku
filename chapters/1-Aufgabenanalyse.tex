\chapter{Aufgabenanalyse}\label{ch:aufgabenanalyse}


\section{Interpretation der Aufgabe}\label{sec:interpretation-der-aufgabe}
Zur Auswertung von Signalen eines optischen Autokorrelators soll ein Programm entwickelt werden.
Der Autokorrelator liefert fortlaufend Daten, die vom Programm eingelesen, verarbeitet und ausgegeben werden.
Weil diese Schritte für unterschiedliche Datensätze unterschiedliche Laufzeiten haben, sollen sie in voneinander unabhängigen Threads laufen.

Die einzelnen Datensätze werden zum Test in zehn unterschiedlichen Textdateien zur Verfügung gestellt.
Diese sind durchnummeriert von null bis neun und haben die folgende Namenskonvention:
\begin{center}
    <nummer>.txt
\end{center}
Ein Datensatz enthält jeweils $N$ Messdaten, die aus zwei positiven Ganzzahlen bestehen.
Die erste Zahl steht dabei für die Intensität des Messsignals und wird im Folgenden als y-Wert interpretiert.
Die zweite Zahl bezeichnet die Spielposition und wird als \~x-Wert interpretiert.
Der Spiegel ist eine Komponente des Autokorrelators die vor und zurück bewegt werden kann, um das Output-Signal zu manipulieren.

Der einlesende Thread soll die zehn Textdateien in chronologischer Reihenfolge einlesen und in eine geeignete Datenstruktur überführen.
Wenn zehnte Datei fertig bearbeitet wurde, soll wieder bei ersten begonnen werden.
Dieses Verhalten wird fortgesetzt bis ein Signal zum Programmabbruch empfangen wird.
Es soll hier das Verhalten im produktiven Betrieb simuliert werden, in dem nicht nur zehen Datensätze bearbeitet werden, sondern fortlaufend neue Datensätze vom Autokorrelator produziert werden.
Der Einlese-Thread stellt jeden Datensatz für 0,05 Sekunden zur Abholung über eine Schnittstelle zur Verfügung.

Der Verarbeitungs-Thread holt sich über diese Schnittstelle Datensätze, führt diverse Transformationen und Berechnungen durch und sendet die Ergebnisse an den Ausgabe-Thread.
Es werden nur Datensätze verarbeitet, die die Verarbeitung vorher noch nicht durchlaufen haben.
Auch die Verarbeitung läuft bis ein Signal zum Programmabbruch empfangen wird.

Gleiches gilt für den Ausgabe-Thread.
Dieser bietet eine Schnittstelle, über die verarbeitete Datensätze angenommen werden.
Die Ergebnisse werden wieder in eine Textdatei geschrieben.
Wenn die Ausgabe eines Datensatzes erfolgreich war, wird das an zentraler Stelle vermerkt.
So kann erkannt werden, wenn alle Eingabedateien erfolgreich verarbeitet und ausgegeben wurden.
Dann ist das Kriterium zum Programmende erreicht und alle Threads können beendet werden.


\section{Eingabeformat}\label{sec:eingabe-format}
Wie bereits beschrieben liegen die Datensätze in Textdateien.
In diesen Dateien sind die einzelnen Messpunkte, bestehend aus x- und y-Wert, zeilenweise zu finden.
An erster Stelle steht der y-Wert.
Getrennt durch einen Tabulator folgt der x-Wert.
Zeilen die mit einem \#-Zeichen beginnen sind Kommentare und können ignoriert werden.
\begin{figure}[htb]
    \centering
    \includegraphics[width=0.5\linewidth]{images/eingabeDat_bsp}
    \caption{
        Beispiel einer Eingabedatei.
    }
    \label{fig:eingabe_dat_beispiel}
\end{figure}


\section{Ausgabeformat}\label{sec:ausgabeformat}
Nach erfolgreicher Verarbeitung werden die Ergebnisse wieder in eine Textdatei geschrieben.
Diese dateien haben den Namen der zugehörigen Eingabedatei mit dem zusätzlichen Präfix \enquote{out}.
Die Eingabedatei \enquote{1.txt} hat dann z.B. \enquote{out1.txt}.

In die erste Zeile werden die Informationen zur berechneten Pulsbreite geschrieben.
Dabei soll das folgende Format verwendet werden:
\begin{center}
    \# FWHM = <Pulsbreite:float>, <Links-Index:int>, <Rechts-Index:int>
\end{center}
In den folgenden Zeilen sollen wieder zeilenweise die verarbeiteten Messpunkte stehen.
Pro Zeile werden, jeweils getrennt durch einen Tabulator, erwartet:
\begin{itemize}[noitemsep]
    \item Der umgeformte und geglättete x-Wert.
    \item Der skalierte y-Wert.
    \item Der Wert der oberen Einhüllenden an der passenden Stelle.
\end{itemize}

\begin{figure}[htb]
    \centering
    \includegraphics[width=\linewidth]{images/ausgabeDat_bsp}
    \caption{
        Beispiel einer Ausgabedatei.
    }
    \label{fig:ausgabe_dat_beispiel}
\end{figure}

\section{Fehlerarten}\label{sec:fehlerarten}

\subsection{Technische Fehler}\label{subsec:technische-fehler}

\subsection{Syntaktische Fehler}\label{subsec:syntaktische-fehler}

\subsection{Semantische Fehler}\label{subsec:semantische-fehler}


\section{Fehlerbehandlung}\label{sec:fehlerbehandlung}

\subsection{Technische Fehler}\label{subsec:technische-fehler-behandlung}

\subsection{Syntaktische Fehler}\label{subsec:syntaktische-fehler-behandlung}

\subsection{Semantische Fehler}\label{subsec:semantische-fehler-behandlung}

\subsection{Sonderfälle}\label{subsec:sonderfaelle}
