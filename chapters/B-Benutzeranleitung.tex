\chapter{Benutzeranleitung}\label{ch:benutzeranleitung}

\section{Verzeichnisstruktur}\label{sec:verzeichnisstruktur}
Im \textbf{Wurzelverzeichnis} sind zu finden:
\begin{itemize}[noitemsep]
    \item Die kompilierte, ausführbare .jar-Datei, mit dem Namen \glqq GroPro-1.0-\\SNAPSHOT.jar\grqq{}.
    \item Das Batch-Skript zur automatischen Ausführung der Tests mit Namen \glqq AutomatischeTests.cmd\grqq{}.
    \item Der Ordner \glqq src\grqq{}, in dem der Source-Code des Java-Projekts zu finden ist.
    \item Der Ordner \glqq TestEingaben\grqq{}, in dem die Beispiel-Eingaben als Text-Dateien liegen.
    \item Der Ordner \glqq TestAusgaben\grqq{}, der beim Ausführen des Programms erstellt wird und die Ergebnisse des Programmdurchlaufs enthält.
    \item Die Dokumentation als PDF-Dokument.
    \item Eine Datei \glqq pom.xml\grqq{}, die zum Bauen der Jar-Datei benötigt wird.
\end{itemize}


\section{Vorbereiten des Systems}\label{sec:vorbereiten-des-systems}

\subsection{Systemvoraussetzungen}\label{subsec:systemvoraussetzungen}
Damit sichergestellt ist, dass das Programm ausführbar ist, sollte Windows 10 (\url{https://www.microsoft.com/de-de/software-download/windows10}) als Betriebssystem verwendet werden.
Zudem muss eine JRE oder JDK mit mindestens Version 8 installiert sein (Vorzugsweise die JDK verwenden, da diese auch zum Kompilieren benötigt wird).
Das Bin-Verzeichnis dieser Installation sollte in den Umgebungsvariablen im Path aufgenommen werden.
Unter \url{https://jdk.java.net/} werden unterschiedliche Versionen der OpenJDK von Oracle zum Download bereitgestellt.

\subsection{Installation}\label{subsec:installation}
Zur Installation muss lediglich die Zip-Datei entpackt werden.

\section{Programmaufruf}\label{sec:programmaufruf}
Nachdem die abgegebene Zip-Datei entpackt wurde, kann das Programm über den Befehl
\begin{center}
    \colorbox{gray!20}{
        \begin{minipage}{0.9\textwidth}
            java -jar GroPro-1.0-SNAPSHOT.jar Beispiel1.txt 1
        \end{minipage}
    }
\end{center}
im CMD-Fenster gestartet werden.
Dabei sind die letzten beiden Parameter der Dateiname des Tests und die Nummer des Tests.

\section{Testen der Beispiele}\label{sec:testen-der-beispiele}
Zum Ausführen der automatischen Tests muss die Datei \glqq AutomatischeTests.cmd \grqq{} mit einem Doppelklick gestartet werden.
Dadurch wird das Programm jeweils mit den, im Ordner \glqq TestEingaben \grqq{} enthaltenen, Dateien aufgerufen.
Die Ergebnisse landen im oben beschriebenen Output-Verzeichnis.

\section{Kompilieren}\label{sec:kompilieren}

Zum Erzeugen der ausführbaren Jar-Datei wird Maven verwendet.
Unter \url{https://maven.apache.org/download.cgi} kann eine Maven-Installation als Zip-Datei heruntergeladen werden (vorzugsweise Version 3.8.5 verwenden).
Auch hier muss das Bin-Verzeichnis der Installation in den Umgebungsvariablen im Path eingetragen werden.
Damit Maven funktioniert, muss zudem eine JDK installiert sein (siehe Abschnitt~\ref{subsec:systemvoraussetzungen}).

Nun kann über Öffnen eines CMD-Fensters im Wurzelverzeichnis das Kommando
\begin{center}
    \colorbox{gray!20}{
        \begin{minipage}{0.9\textwidth}
            mvn package
        \end{minipage}
    }
\end{center}
verwendet werden.
Die ausführbare Jar-Datei befindet sich dann im Ordner \glqq target\grqq{}.